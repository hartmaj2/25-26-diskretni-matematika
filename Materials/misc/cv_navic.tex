% Document class
\documentclass[10pt]{article}

% Preamble
\usepackage[a5paper,margin=1cm,includefoot]{geometry} % includefoot makes footer stay inside of the page
\usepackage{titling}
\usepackage{fancyhdr}
\usepackage[hidelinks]{hyperref}
\usepackage{amsthm}
\usepackage{mathtools}
\usepackage{enumitem} % for a,b,c enums

%% metadata settings
\title{Příklady navíc}
\author{Jan Hartman}
\date{16.9.2025}

\newcommand{\teacherurl}{https://kam.mff.cuni.cz/~hartmaj/}
\newcommand{\titlerule}{%
    \noindent %
    \makebox[\textwidth]{\large \thetitle \hfill \thedate}
    \rule{\textwidth}{0.4pt}%
}


%% headers and footers settings
\renewcommand{\headrulewidth}{0pt}
\renewcommand{\footrulewidth}{0.4pt}
\pagestyle{fancy}
\fancyhf{} % clear all header/footer (e.g. by default, footer would show page numbers)
% \fancyhead[L]{\thetitle}   % left -> title
% \fancyhead[C]{} % center -> unused
% \fancyhead[R]{\thedate}  % right -> date
\fancyfoot[C]{\small Více info k cvičení: \url{\teacherurl}}  


% theorem styles
\newtheoremstyle{definitionstyle}{10pt}{10pt}{\normalfont}{}{\bfseries}{.}{ }{\thmname{#1}}
\newtheoremstyle{problemstyle}{10pt}{10pt}{\normalfont}{}{\bfseries}{.\newline}{ }{\thmname{#1}\thmnumber{ #2}\thmnote{ (#3)}}

% theorems
\theoremstyle{definitionstyle}
\newtheorem{defn}{Definice}
\theoremstyle{problemstyle}
\newtheorem{problem}{Příklad}

% Document body
\begin{document}

\titlerule

\begin{problem}[Hazard s truhličkami]
V casinu ti nabídli následující hru: Nejprve zaplatíš 1 euro za vstup do hry. Dále je před tebe položeno 6 truhliček, kde v každé leží jeden cent. Můžeš ve kterémkoliv tahu vyjmout obsah všech truhliček, takto získanou částku si ponechat a odejít ze hry ven. Dále můžeš zahodit jeden cent z nějaké přihrádky a vložit dva centy do přihrády bezprostředně napravo od ní.

\begin{enumerate}[label=(\alph*)]
\item Vyplatí se ti přijmout nabídku a hrát?
\item Máš k dispozici navíc akci: Zahodit cent z nějaké přihrádky a prohodit obsah dvou přihrádek bezprostředně vpravo od této přihrádky. Kolik peněz zvládneš vydělat nyní?
\end{enumerate}

\begin{problem}
Definujeme spojku $\downarrow$ následovně: Pro výroky $a$,$b$ je $a \downarrow b$ pravdivý právě tehdy když $a$ i $b$ jsou oba lživé. Vyjádřete logické operátory $\neg, \vee, \wedge, \Rightarrow$ jen za použití spojky $\downarrow$. 
\end{problem}

\end{problem}

\end{document}