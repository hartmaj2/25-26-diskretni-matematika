% Document class
\documentclass[10pt]{article}

% Preamble
\usepackage[a5paper,margin=1cm,includefoot]{geometry} % includefoot makes footer stay inside of the page

\usepackage{../../tutorial}


%% metadata settings
\newcommand{\tutnum}{4}
\title{\tutnum. písemka z diskrétní matematiky}
\author{Jan Hartman}

\newcommand{\teacherurl}{https://kam.mff.cuni.cz/~hartmaj/}
\newcommand{\titlerule}{%
    \noindent %
    \makebox[\textwidth]{\large \thetitle \hfill Jméno: \hspace{4cm}}
    \rule{\textwidth}{0.4pt}%
}

%% headers and footers settings
\renewcommand{\headrulewidth}{0pt}
\renewcommand{\footrulewidth}{0.4pt}
\pagestyle{fancy}
\fancyhf{} % clear all header/footer (e.g. by default, footer would show page numbers)
% \fancyhead[L]{\thetitle}   % left -> title
% \fancyhead[C]{} % center -> unused
% \fancyhead[R]{\thedate}  % right -> date
\fancyfoot[C]{\small Nezapomeňte vaše výsledky zdůvodnit!}  


% Document body
\begin{document}

\titlerule

\begin{problem}[4 body]
Napište definice minimálního a nejmenšího prvku vzhledem k uspořádání $\preceq$ na množině $X$. Dbejte na správnou matematickou notaci.

\end{problem}

\begin{problem}[6 bodů]
Nechť $R$ je relace na množině $\mathbb{Z}$ t.ž. $aRb$ právě tehdy když $|a-b| \leq 1$. Určete, které z vlastností reflexivita, symetrie, antisymetrie a tranzitivita tato relace splňuje. Dle vašeho výsledku určete, zda se jedná o ekvivalenci, uspořádání, či ani jedno.
\end{problem}
\end{document}