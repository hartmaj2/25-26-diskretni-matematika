% Document class
\documentclass[10pt]{article}

% Preamble
\usepackage[a5paper,margin=1cm,includefoot]{geometry} % includefoot makes footer stay inside of the page

\usepackage{../../tutorial}


%% metadata settings
\newcommand{\tutnum}{10}
\title{\tutnum. písemka z diskrétní matematiky}
\author{Jan Hartman}

\newcommand{\teacherurl}{https://kam.mff.cuni.cz/~hartmaj/}
\newcommand{\titlerule}{%
    \noindent %
    \makebox[\textwidth]{\large \thetitle \hfill Jméno: \hspace{4cm}}
    \rule{\textwidth}{0.4pt}%
}

%% headers and footers settings
\renewcommand{\headrulewidth}{0pt}
\renewcommand{\footrulewidth}{0.4pt}
\pagestyle{fancy}
\fancyhf{} % clear all header/footer (e.g. by default, footer would show page numbers)
% \fancyhead[L]{\thetitle}   % left -> title
% \fancyhead[C]{} % center -> unused
% \fancyhead[R]{\thedate}  % right -> date
\fancyfoot[C]{\small Nezapomeňte vaše výsledky zdůvodnit!}  


% Document body
\begin{document}

\titlerule

\begin{problem}[5 bodů]
Nechť $T$ je strom na $10$ vrcholech, který obsahuje právě $3$ listy. Určete počet vrcholů stupně $2$ v $T$.
\end{problem}

\begin{problem}[5 bodů]
Určete počet koster následujících grafů:
\begin{enumerate}[label=\alph*)]
    \item $K_4$, tedy úplného grafu na čtyřech vrcholech
    \item grafu na obrázku níže
\end{enumerate}

\begin{figure}[h]
    \centering
    \includegraphics{kostry.pdf}
\end{figure}

\end{problem}

\begin{problem}[2 bonusové body]
Kolik mají grafy z předchozích úloh navzájem neizomorfních koster.
\end{problem}


\end{document}