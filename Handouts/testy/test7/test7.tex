% Document class
\documentclass[10pt]{article}

% Preamble
\usepackage[a5paper,margin=1cm,includefoot]{geometry} % includefoot makes footer stay inside of the page

\usepackage{../../tutorial}


%% metadata settings
\newcommand{\tutnum}{7}
\title{\tutnum. písemka z diskrétní matematiky}
\author{Jan Hartman}

\newcommand{\teacherurl}{https://kam.mff.cuni.cz/~hartmaj/}
\newcommand{\titlerule}{%
    \noindent %
    \makebox[\textwidth]{\large \thetitle \hfill Jméno: \hspace{4cm}}
    \rule{\textwidth}{0.4pt}%
}

%% headers and footers settings
\renewcommand{\headrulewidth}{0pt}
\renewcommand{\footrulewidth}{0.4pt}
\pagestyle{fancy}
\fancyhf{} % clear all header/footer (e.g. by default, footer would show page numbers)
% \fancyhead[L]{\thetitle}   % left -> title
% \fancyhead[C]{} % center -> unused
% \fancyhead[R]{\thedate}  % right -> date
\fancyfoot[C]{\small Nezapomeňte vaše výsledky zdůvodnit!}  


% Document body
\begin{document}

\titlerule

\begin{problem}[5 bodů]
Nechť $(\Omega,2^{\Omega},P)$ je diskrétní pravděpodobnostní prostor a $A,B \in 2^{\Omega}$ jsou jevy.
\begin{enumerate}[label=\alph*)]
    \item Napište definici nezávislosti jevů $A$ a $B$.
    \item Pokud jsou $A$ a $B$ nezávislé, víme něco o $P(A \mid B)$? Zdůvodněte.
\end{enumerate}
\end{problem}

\begin{problem}[5 bodů]
Ve vaku máme 10 skleněných a 20 keramických sfér. Vytáhneme náhodných 7 z nich.
Jaká je pravděpodobnost, že v těchto 7 sférách jsou právě 3 skleněné?
(sféry do vaku nevracíme, výsledek není třeba vyčíslit).
\end{problem}

\begin{problem}[2 bonusové body]
Popište formálně množinu elementárních jevů z předchozí úlohy.
\end{problem}

\end{document}