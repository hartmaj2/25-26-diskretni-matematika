% Document class
\documentclass[10pt]{article}

% Preamble
\usepackage[a5paper,margin=1cm,includefoot]{geometry} % includefoot makes footer stay inside of the page

\usepackage{../../tutorial}


%% metadata settings
\newcommand{\tutnum}{2}
\title{\tutnum. písemka z diskrétní matematiky}
\author{Jan Hartman}

\newcommand{\teacherurl}{https://kam.mff.cuni.cz/~hartmaj/}
\newcommand{\titlerule}{%
    \noindent %
    \makebox[\textwidth]{\large \thetitle \hfill Jméno: \hspace{4cm}}
    \rule{\textwidth}{0.4pt}%
}

%% headers and footers settings
\renewcommand{\headrulewidth}{0pt}
\renewcommand{\footrulewidth}{0.4pt}
\pagestyle{fancy}
\fancyhf{} % clear all header/footer (e.g. by default, footer would show page numbers)
% \fancyhead[L]{\thetitle}   % left -> title
% \fancyhead[C]{} % center -> unused
% \fancyhead[R]{\thedate}  % right -> date
\fancyfoot[C]{\small Nezapomeňte vaše výsledky zdůvodnit!}  


% Document body
\begin{document}

\titlerule

\begin{problem}[4 body]
Dokažte indukcí, že pro každé $n \geq 1$ platí $\sum_{i=1}^{n}(2i-1) = n^2$
\end{problem}

\begin{problem}[6 bodů]

Dokaže indukcí, že pro každé liché $n \geq 1$ platí $8 \ | \ (n^2 - 1)$. Jinými slovy pro každé $n \in \{1,3,5,7,\ldots\}$. \textit{Hint: Když si už nebudete vědět rady, využijte definici lichosti.}

\end{problem}

\end{document}
