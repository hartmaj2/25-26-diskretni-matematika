% Document class
\documentclass[10pt]{article}

% Preamble
\usepackage[a5paper,margin=1cm,includefoot]{geometry} % includefoot makes footer stay inside of the page

\usepackage{../../tutorial}


%% metadata settings
\newcommand{\tutnum}{8}
\title{\tutnum. písemka z diskrétní matematiky}
\author{Jan Hartman}

\newcommand{\teacherurl}{https://kam.mff.cuni.cz/~hartmaj/}
\newcommand{\titlerule}{%
    \noindent %
    \makebox[\textwidth]{\large \thetitle \hfill Jméno: \hspace{4cm}}
    \rule{\textwidth}{0.4pt}%
}

%% headers and footers settings
\renewcommand{\headrulewidth}{0pt}
\renewcommand{\footrulewidth}{0.4pt}
\pagestyle{fancy}
\fancyhf{} % clear all header/footer (e.g. by default, footer would show page numbers)
% \fancyhead[L]{\thetitle}   % left -> title
% \fancyhead[C]{} % center -> unused
% \fancyhead[R]{\thedate}  % right -> date
\fancyfoot[C]{\small Nezapomeňte vaše výsledky zdůvodnit!}  


% Document body
\begin{document}

\titlerule

\begin{problem}[5 bodů]
Nechť $(\Omega,P)$ je diskrétní pravděpodobnostní prostor a $X : \Omega \rightarrow \R$ je náhodná veličina. 

\begin{enumerate}[label=\alph*)]
    \item Definujte střední hodnotu náhodné veličiny $X$.
    \item Nechť $Y : \Omega \rightarrow \R$ je druhá náhodná veličina. Co víme o střední hodnotě náhodné veličiny $X + Y$?
\end{enumerate}
\end{problem}

\begin{problem}[5 bodů]
Máme $n$ dětí a $m$ hraček. Každé dítě si náhodně zvolí právě jednu hračku a k té se rozběhne. O libovolné dvojici dětí $\{i,j\}$ pro $i \neq j$ řekneme, že je konfliktní, pokud se $i$ a $j$ rozběhly ke stejné hračce. Jaká je střední hodnota počtu konfliktních dvojic dětí?
\end{problem}

\end{document}