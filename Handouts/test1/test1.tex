% Document class
\documentclass[10pt]{article}

% Preamble
\usepackage[a5paper,margin=1cm,includefoot]{geometry} % includefoot makes footer stay inside of the page
\usepackage{titling}
\usepackage{fancyhdr}
\usepackage[hidelinks]{hyperref}
\usepackage[czech]{babel}
\usepackage{amsthm}
\usepackage{mathtools}
\usepackage{enumitem} % for a,b,c enums
\usepackage{caption}

% FONT SPEC (for XeLaTeX) TO MAKE TROJA PRINTER WORK
\usepackage{fontspec}
\usepackage{unicode-math}


%% metadata settings
\newcommand{\tutnum}{1}
\title{\tutnum. písemka z diskrétní matematiky}
\author{Jan Hartman}

\newcommand{\teacherurl}{https://kam.mff.cuni.cz/~hartmaj/}
\newcommand{\titlerule}{%
    \noindent %
    \makebox[\textwidth]{\large \thetitle \hfill Jméno: \hspace{3cm}}
    \rule{\textwidth}{0.4pt}%
}

%% headers and footers settings
\renewcommand{\headrulewidth}{0pt}
\renewcommand{\footrulewidth}{0.4pt}
\pagestyle{fancy}
\fancyhf{} % clear all header/footer (e.g. by default, footer would show page numbers)
% \fancyhead[L]{\thetitle}   % left -> title
% \fancyhead[C]{} % center -> unused
% \fancyhead[R]{\thedate}  % right -> date
\fancyfoot[C]{\small Nezapomeňte uvádět postup!}  


% theorem styles
\newtheoremstyle{definitionstyle}{10pt}{10pt}{\normalfont}{}{\bfseries}{.}{ }{\thmname{#1}}
\newtheoremstyle{problemstyle}{10pt}{10pt}{\normalfont}{}{\bfseries}{:}{ }{\thmname{#1}\thmnumber{ #2}\thmnote{ (#3)}}

% theorems
\theoremstyle{definitionstyle}
\newtheorem{defn}{Definice}
\theoremstyle{problemstyle}
\newtheorem{problem}{Příklad}

% Document body
\begin{document}

\titlerule

\begin{problem}[5 bodů]
Zjednodušte výraz $A \setminus ( B \setminus ( A \setminus ( B \setminus ( A \setminus ( B \setminus A )))))$ na tvar, ve kterém se symboly $A$, $B$ a $\setminus$ vyskytují nejvýše jednou.
\end{problem}

\begin{problem}[5 bodů]
Zjednodušte výraz $((p \wedge q) \vee (\neg p \wedge q) ) \vee ( (p \wedge \neg q) \vee (p \wedge q)  )$ na tvar, ve kterém se symboly $p,q,\neg,\wedge,\vee$ vyskytují nejvýše jednou.
\end{problem}

% \begin{problem}
% Definujeme spojku $\downarrow$ následovně: Pro výroky $a$,$b$ je $a \downarrow b$ pravdivý právě tehdy když $a$ i $b$ jsou oba lživé. Vyjádřete logické operátory $\neg, \vee, \wedge, \Rightarrow$ jen za použití spojky $\downarrow$. 
% \end{problem}

\end{document}
