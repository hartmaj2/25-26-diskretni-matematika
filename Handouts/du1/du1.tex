% Document class
\documentclass[11pt]{article}

% Preamble
\usepackage[a4paper,margin=2.5cm,includefoot]{geometry} % includefoot makes footer stay inside of the page
\usepackage{titling}
\usepackage{fancyhdr}
\usepackage[hidelinks]{hyperref}
\usepackage{amsthm}
\usepackage{mathtools}
\usepackage[czech]{babel}

%% metadata settings
\newcommand{\tutnum}{1}
\title{\tutnum. domácí úkol z diskrétní matematiky}
\author{Jan Hartman}
\date{6.10.2025}

\newcommand{\teacherurl}{https://kam.mff.cuni.cz/~hartmaj/}
\newcommand{\titlerule}{%
    \noindent %
    \makebox[\textwidth]{\large \thetitle \hfill termín: \thedate}
    \rule{\textwidth}{0.4pt}%
}


%% headers and footers settings
\renewcommand{\headrulewidth}{0pt}
\renewcommand{\footrulewidth}{0.4pt}
\pagestyle{fancy}
\fancyhf{} % clear all header/footer (e.g. by default, footer would show page numbers)
% \fancyhead[L]{\thetitle}   % left -> title
% \fancyhead[C]{} % center -> unused
% \fancyhead[R]{\thedate}  % right -> date
\fancyfoot[C]{\small Více info k cvičení: \url{\teacherurl}}  


% theorem styles
\newtheoremstyle{definitionstyle}{10pt}{10pt}{\normalfont}{}{\bfseries}{.}{ }{\thmname{#1}}
\newtheoremstyle{problemstyle}{10pt}{10pt}{\normalfont}{}{\bfseries}{.\newline}{ }{\thmname{#1}\thmnumber{ #2}\thmnote{ (#3)}}

% theorems
\theoremstyle{definitionstyle}
\newtheorem{defn}{Definice}
\theoremstyle{problemstyle}
\newtheorem{problem}{Příklad}

% Document body
\begin{document}

\titlerule

\begin{problem}[Nemáme rádi kvantifikátory]
Uvažme množinu čísel $M = \{ 0,1,2,3 \}$ a označme $S(x,y)$ výrok $y = x + 1$. Rozepište výrok $\exists z \in M : S(0,z)$ bez použití kvantifikátorů a bez značení $S(x,y)$. Následně vyhodnoťte, zda výrok platí či nikoliv.
\end{problem}

\begin{problem}[Množinový průnik na více způsobů]
Vyjádřete $A \cap B$ pouze za pomocí množinového rozdílu $\setminus$ a vhodného uzávorkování. Množiny $A,B$ i operace $\setminus$ se mohou ve výrazu vyskytovat libovolněkrát. Nesmíte však použít \textbf{žádnou} jinou množinovou operaci.
\end{problem}

\end{document}