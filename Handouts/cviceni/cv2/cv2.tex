% Document class
\documentclass[10pt]{article}

% Preamble
\usepackage[a4paper,margin=2.5cm,includefoot]{geometry} % includefoot makes footer stay inside of the page

\usepackage{../../tutorial}

%% metadata settings
\newcommand{\tutnum}{2}
\title{\tutnum. cvičení z diskrétní matematiky}
\author{Jan Hartman}
\date{13.10.2025}

\newcommand{\teacherurl}{https://kam.mff.cuni.cz/~hartmaj/}
\newcommand{\titlerule}{%
    \noindent %
    \makebox[\textwidth]{\large \thetitle \hfill \thedate}
    \rule{\textwidth}{0.4pt}%
}

%% headers and footers settings
\renewcommand{\headrulewidth}{0pt}
\renewcommand{\footrulewidth}{0.4pt}
\pagestyle{fancy}
\fancyhf{} % clear all header/footer (e.g. by default, footer would show page numbers)
% \fancyhead[L]{\thetitle}   % left -> title
% \fancyhead[C]{} % center -> unused
% \fancyhead[R]{\thedate}  % right -> date
\fancyfoot[C]{\small Více info k cvičení: \url{\teacherurl}}  

% Document body
\begin{document}

\titlerule

\begin{defn}
Definujeme $d = \gcd(a,b)$ jako největší takové číslo, pro které platí $d \mid a$ a zároveň $d \mid b$.
\end{defn}

\begin{problem}[Relace podobojí]
Najděte relaci na $\{1,2,3,4\}$, která je zároveň symetrická i antisymetrická. Kolik takových relací existuje?
\end{problem}

\begin{problem}[Relace podžádnou]
Najděte relaci na $\{1,2,3,4\}$, která není ani symetrická, ani antisymetrická.
\end{problem}

\begin{problem}[Jaké mají vlastnosti?]
Rozhodněte, které z následujících relací jsou reflexivní, symetrické, tranzitivní nebo antisymetrické:
\begin{enumerate}[label=(\alph*)]
    \item Relace $R=\{(1,1),(1,2),(2,1),(2,2),(3,3)\}$ na množině $\{1,2,3\}$
    \item Relace $R=\{(1,1),(1,2),(2,1),(3,3)\}$ na množině $\{1,2,3\}$
    \item Relace $\leq$ na množině $\mathbb{N}$
    \item Relace $R = \{ (x,y) : \gcd(x,y) = 1 \}$ na množině $\{1,2,3,4,5\}$
\end{enumerate}
\end{problem}

\begin{problem}[Kolik jich je?]
Určete, kolik je na $n$-prvkové množině relací:
\begin{enumerate}[label=(\alph*)]
    \item všech možných
    \item reflexivních
    \item symetrických
    \item antisymetrických
\end{enumerate}
\end{problem}

\begin{problem}[Prostá $\times$ na]
Dokaže, že máme-li konečnou množinu $A$, tak je funkce $f : A \rightarrow A$ prostá, právě tehdy když je na. Platí to i pro nekonečnou $A$?
\end{problem}

\begin{problem}[Složená relace]
Jak vypadá relace $R \circ R$, je-li $R$ definovaná jako:
\begin{enumerate}[label=(\alph*)]
    \item relace $=$ na $\mathbb{N}$
    \item relace $\leq$ na $\mathbb{N}$
    \item relace $<$ na $\mathbb{N}$
    \item relace $<$ na $\mathbb{R}$
\end{enumerate}
\end{problem}

\begin{problem}[Nekomutativita skládání]
Najděte relace $R,S$ na libovolné množině $X$ takové, že $R \circ S \neq S \circ R$.
\end{problem}

\begin{problem}[Zachovávání vlastností]
Nechť $X$ je konečná množina a $R,S$ jsou relace na této množině. 

\noindent
Rozhodněte, zda pro $V \in \{\text{reflexivní}, \text{symetrická}, \text{antisymetrická},\text{tranzitivní}\}$ a pro $\square \in \{\cap,\cup,\setminus,\circ\}$ platí: „Když mají $R$ i $S$ vlastnost $V$, tak nutně i $R \ \square \ S$ má vlastnost $V$.“.
\end{problem}

\begin{problem}[Skládání funkcí]
Rozhodněte o funkcích $f: X \rightarrow Y$ a $g: Y \rightarrow Z$ pro libovolné množiny $X,Y,Z$:
\begin{enumerate}[label=(\alph*)]
    \item Jsou-li $f,g$ prosté funkce, musí nutně i $g \circ f$ být prostá?
    \item Jsou-li $f,g$ funkce na, musí nutně i $g \circ f$ být na?
    \item Je-li $g \circ f$ prostá, musí být $f$ nebo $g$ prostá?
    \item Je-li $g \circ f$ na, musí být $f$ nebo $g$ na?
\end{enumerate}
\end{problem}

\end{document}