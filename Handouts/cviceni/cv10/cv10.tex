% Document class
\documentclass[10pt]{article}

% Preamble
\usepackage[a4paper,margin=2.5cm,includefoot]{geometry} % includefoot makes footer stay inside of the page

\usepackage{../../tutorial}
\usepackage{subcaption}

%% metadata settings
\newcommand{\tutnum}{10}
\title{\tutnum. cvičení z diskrétní matematiky}
\author{Jan Hartman}
\date{8.12.2025}

\newcommand{\teacherurl}{https://kam.mff.cuni.cz/~hartmaj/}
\newcommand{\titlerule}{%
    \noindent %
    \makebox[\textwidth]{\large \thetitle \hfill \thedate}
    \rule{\textwidth}{0.4pt}%
}

%% headers and footers settings
\renewcommand{\headrulewidth}{0pt}
\renewcommand{\footrulewidth}{0.4pt}
\pagestyle{fancy}
\fancyhf{} % clear all header/footer (e.g. by default, footer would show page numbers)
% \fancyhead[L]{\thetitle}   % left -> title
% \fancyhead[C]{} % center -> unused
% \fancyhead[R]{\thedate}  % right -> date
\fancyfoot[C]{\small Více info k cvičení: \url{\teacherurl}}  

% Napady:
% Nejmensi graf t.z. splnuje eulerovu formuli, ale neni to strom
% Splňuje-li graf Eulerovu formuli, tak obsahuje list nebo izolovaný vrchol

% Definovat:
% strom
% kostra
% vnitrni vrchol, list

% Document body
\begin{document}

\titlerule

\begin{problem}[Co se pozná ze skóre]
Zdůvodněte jaké z následujících vlastností grafu lze vyčíst z jeho skóre:
\begin{enumerate}[label=\alph*)]
    \item počet jeho vrcholů
    \item počet jeho hran
    \item počet jeho komponent souvislosti
    \item zda je to strom či nikoliv
\end{enumerate}
\end{problem}

\begin{problem}[Košaté stromy]
Dokažte, že obsahuje-li strom vrchol stupně $k$, pak v něm je alespoň $k$ listů.
\end{problem}

\begin{problem}[Lekce anatomie]
Dokažte, že když je graf souvislý, tak má kostru. Platí to i naopak?
\end{problem}

\begin{problem}[Ještě jedna definice stromu]
Dokažte, že graf je strom právě tehdy, když je acyklický a platí Eulerova formule $|E| = |V| - 1$.
\end{problem}

\begin{problem}[Lesní formule]
Ukažte, že platí: Graf s $k$ komponentami souvislosti je les právě tehdy, když platí $|E| = |V| - k$.
\end{problem}

\begin{problem}[Most, to je hlavní kost]
Ukažte, že je-li hrana v grafu most, pak ji každá kostra obsahuje.
\end{problem}

\begin{problem}[Kolik koster si přejete, prosím?]
Pro která $k \ge 0$ existuje graf s právě $k$ kostrami?
\end{problem}

\begin{problem}[Průměrný stupeň]
Ukažte, že pokud pro graf $G$ platí Eulerova formule, tak obsahuje izolovaný vrchol nebo list.
\end{problem}

\begin{problem}[Kostra sem, kostra tam]
Nechť $m$, $n$ a $k$ jsou přirozená čísla taková, aby příslušná zadání dávala smysl. Určete počet koster následujících grafů:
\begin{enumerate}[label=\alph*)]
    \item strom na $n$ vrcholech
    \item úplný bipartitní graf $K_{n,2}$
    \item $\Theta$-graf, tedy dva vrcholy stupně 3 spojené třemi cestami o délkách $m,n,k$. (Množiny vnitřních vrcholů jednotlivých cest jsou navzájem disjunktní)
    % \item Úplný bipartitní graf $K_{n,3}$
    \item „činka“, tedy dvě kružnice $C_m, C_n$, kde je jeden vrchol první kružnice spojen s vrcholem druhé kružnice cestou délky $k$
    \item úplný graf na čtyřech vrcholech $K_4$
\end{enumerate}
\end{problem}

\begin{problem}[Nadvláda listů]
Ukažte, že každý netriviální strom, který neobsahuje vrchol stupně $2$, má více listů než vnitřních vrcholů.
\end{problem}

\begin{problem}[Extrémní grafy]
Kolik nejvíce a kolik nejméně hran může mít:
\begin{enumerate}[label=\alph*)]
    \item souvislý graf na $n$ vrcholech
    \item acyklický graf
    \item graf s $k$ komponentami souvislosti
    % \item Úplný bipartitní graf $K_{n,3}$
    \item graf, který neobsahuje trojúhelník $C_3$ jako podgraf
\end{enumerate}
\end{problem}


\end{document}
