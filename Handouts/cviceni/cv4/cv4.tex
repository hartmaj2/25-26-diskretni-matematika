% Document class
\documentclass[10pt]{article}

% Preamble
\usepackage[a4paper,margin=2.5cm,includefoot]{geometry} % includefoot makes footer stay inside of the page

\usepackage{../../tutorial}
\usepackage{subcaption}

%% metadata settings
\newcommand{\tutnum}{4}
\title{\tutnum. cvičení z diskrétní matematiky}
\author{Jan Hartman}
\date{27.10.2025}

\newcommand{\teacherurl}{https://kam.mff.cuni.cz/~hartmaj/}
\newcommand{\titlerule}{%
    \noindent %
    \makebox[\textwidth]{\large \thetitle \hfill \thedate}
    \rule{\textwidth}{0.4pt}%
}

%% headers and footers settings
\renewcommand{\headrulewidth}{0pt}
\renewcommand{\footrulewidth}{0.4pt}
\pagestyle{fancy}
\fancyhf{} % clear all header/footer (e.g. by default, footer would show page numbers)
% \fancyhead[L]{\thetitle}   % left -> title
% \fancyhead[C]{} % center -> unused
% \fancyhead[R]{\thedate}  % right -> date
\fancyfoot[C]{\small Více info k cvičení: \url{\teacherurl}}  

% Document body
\begin{document}

\titlerule

\begin{problem}[Počty funkcí]
Kolik existuje funkcí z $\{1,2,\ldots,a\}$ do $\{1,2,\ldots,b\}$?
\begin{enumerate}[label=\alph*)]
    \item všech
    \item prostých
    \item bijekcí
    \item \textit{na} pro $a = 4$ a $b=3$ 
\end{enumerate}
\end{problem}

\begin{problem}[Staří známí]
Kolika způsoby lze z $n$ rozlišitelných kuliček vybrat uspořádanou $k$-tici? A kolika neuspořádanou? Co když jsou kuličky nerozlišitelné?
\end{problem}

\begin{problem}[Uvažujeme kombinatoricky]
Dokažte kombinatorickou úvahou:
\begin{enumerate}[label=\alph*)]
    \item $\binom{n}{k} = \binom{n}{n-k}$
    \item $\binom{n}{k} = \binom{n-1}{k-1} + \binom{n-1}{k}$
    \item $\sum_{k=0}^{n}\binom{n}{k} = 2^n$
    \item $\sum_{k=0}^{n}(-1)^k\binom{n}{k} = 0$
\end{enumerate}
\end{problem}

\begin{problem}[Rozlož a spočítej]
Kolik existuje různých ekvivalencí na čtyřech prvcích? A kolik na $n$ prvcích? Stačí najít rekurentní vzorec.
\end{problem}

\begin{problem}[Kuličky a přihrádky]
Kolik existuje možností, jak rozmístit $n$ nerozlišitelných kuliček do $k$ rozlišitelných přihrádek? Co když žádná přihrádka nesmí být prázdná? Co když jsou kuličky rozlišitelné?
\end{problem}

\begin{problem}[Klasifikace]
Z $n$ předmětů vybíráme $k$ předmětů. Do následující tabulky napište počty možných výběrů: \\
\textit{Hint: Zkuste použít řešení příkladů výše.}

\vspace{5pt}
\begin{tabular}{|c|c|c|}
\hline
    Výběry & Záleží na pořadí (variace) & Nezáleží na pořadí (kombinace) \\
\hline
    Bez opakování &  &  \\
\hline
    S opakováním &  &  \\
\hline
\end{tabular}

\end{problem}

\begin{problem}[Sestavme si vládu]
Kombinatorickou úvahou vyjádřete výraz $\sum_{k=1}^{n}k\binom{n}{k}$ bez použití sumy.
\end{problem}

\begin{problem}[Šach mat]
Kolika způsoby lze rozestavit bílého a černého krále na šachovnici tak, aby nestáli na stejném políčku? A kolika způsoby tak, aby se neohrožovali?
\end{problem}

\begin{problem}[Umět si vybrat]
Kolika způsoby lze vybrat množiny $A,B \subseteq [n]$ takové, že:
\begin{enumerate}[label=\alph*)]
    \item $A \subseteq B$
    \item $A = \{x\}$ a $x \in B$
\end{enumerate}
\end{problem}

\begin{problem}[Sportem ku zdraví]
Na hřišti se $12$ kamarádů rozhodlo zahrát si volejbal. Jeden z nich však v průběhu dne musel odejít dodělávat úkoly z diskrétní matematiky. Jak se změní počet způsobů rozdělení do dvou týmů z původních $6$ vs $6$ na výsledných $6$ vs $5$. Bude takový počet poloviční, výrazně menší, nebo dokonce stejný?
\end{problem}

\end{document}