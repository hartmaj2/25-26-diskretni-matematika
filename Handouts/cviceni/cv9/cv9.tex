% Document class
\documentclass[10pt]{article}

% Preamble
\usepackage[a4paper,margin=2.5cm,includefoot]{geometry} % includefoot makes footer stay inside of the page

\usepackage{../../tutorial}
\usepackage{subcaption}

%% metadata settings
\newcommand{\tutnum}{9}
\title{\tutnum. cvičení z diskrétní matematiky}
\author{Jan Hartman}
\date{1.12.2025}

\newcommand{\teacherurl}{https://kam.mff.cuni.cz/~hartmaj/}
\newcommand{\titlerule}{%
    \noindent %
    \makebox[\textwidth]{\large \thetitle \hfill \thedate}
    \rule{\textwidth}{0.4pt}%
}

%% headers and footers settings
\renewcommand{\headrulewidth}{0pt}
\renewcommand{\footrulewidth}{0.4pt}
\pagestyle{fancy}
\fancyhf{} % clear all header/footer (e.g. by default, footer would show page numbers)
% \fancyhead[L]{\thetitle}   % left -> title
% \fancyhead[C]{} % center -> unused
% \fancyhead[R]{\thedate}  % right -> date
\fancyfoot[C]{\small Více info k cvičení: \url{\teacherurl}}  

% Document body
\begin{document}

\titlerule

\begin{defn}
Doplněk grafu $G = (V,E)$ je $\overline{G} = (V, \binom{V}{2} \setminus E)$, tedy graf, který má přesně opačné hrany.
\end{defn}

\begin{defn}
Mějme graf $G=(\{v_1,\ldots,v_n\},E)$. Posloupnost $(\deg(v_1),\ldots,\deg(v_n))$ nazveme \textit{skóre} grafu $G$.
\end{defn}

\begin{problem}[Odhalujeme kamufláž]
Uvažujme „standardní“ grafy $K_n$, $P_n$, $C_n$ a $K_{m,n}$ z přednášky. Které z nich jsou izomorfní?
\end{problem}

\begin{problem}[Já a já se perfektně doplňujeme]
Najděte nějaký graf, který je izomorfní svému doplňku. Co musí platit pro jeho počet vrcholů a hran?
\end{problem}

\begin{problem}[Tudy most nevede]
Dokažte, že graf $G$, jehož každý vrchol má sudý stupeň, neobsahuje most. Jako most považujeme hranu $e$ takovou, že $G - e$ má více komponent souvislosti než $G$.
\end{problem}

\begin{problem}[Stejné skóre není vše]
Najděte dvojici neizomorfních grafů, jež mají stejné skóre:
\begin{enumerate}[label=\alph*)]
    \item libovolnou,
    \item kde oba grafy jsou souvislé,
    \item a navíc mají co nejmenší počet vrcholů.
\end{enumerate}
\end{problem}

\begin{problem}[Někdy skóre není nic]
Rozhodněte, zda existuje graf, jehož skóre je:
\begin{enumerate}[label=\alph*)]
    \item $(2,2,2,2,2,2)$
    \item $(3,3,3,3,3,3,3,3,3,3,3)$
    \item $(0,1,2,3,\dots,n-1)$
\end{enumerate}
\end{problem}

\begin{problem}[Cesta tam a zase zpátky]
Nechť $G$ je souvislý graf, který neobsahuje žádný cyklus a $u,v$ dva jeho vrcholy, které nejsou spojené hranou. Ukažte, že $G + \{u,v\}$ již cyklus obsahovat bude.
\end{problem}

\begin{problem}[Skóre z jedniček či dvojek]
Jak vypadají grafy, které obsahují jen vrcholy stupně:
\begin{enumerate}[label=\alph*)]
    \item $1$,
    \item $2$,
    \item $1$ nebo $2$.
\end{enumerate}
\end{problem}

\begin{problem}[Doplněk nesouvislého]
Dokažte, že doplněk nesouvislého grafu je souvislý (ale opačně to obecně neplatí).
\end{problem}

\begin{problem}[Bipartitní grafy]
Ukažte, že graf je bipartitní právě tehdy, neobsahuje-li kružnici liché délky.
\end{problem}

\begin{problem}[Nedokonalý izomorfismus]
Připomeňme si definici izomorfismu:
\[
G \cong H \;\;\Longleftrightarrow\;\; (\exists f : G \to H \text{ bijekce} : \forall u,v \in V(G)\!: \{u,v\} \in E(G) \iff \{f(u),f(v)\} \in E(G)).
\]

Co se stane, když:
\begin{enumerate}[label=\alph*)]
    \item místo bijekce vyžadujeme jen prostou funkci?
    \item místo bijekce si vystačíme s jakoukoliv funkcí?
    \item místo druhé ekvivalence uvažujeme implikaci?
\end{enumerate}
\end{problem}

\end{document}