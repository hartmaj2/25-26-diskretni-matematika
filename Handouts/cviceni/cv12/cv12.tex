% Document class
\documentclass[10pt]{article}

% Preamble
\usepackage[a4paper,margin=2.5cm,includefoot]{geometry} % includefoot makes footer stay inside of the page

\usepackage{../../tutorial}
\usepackage{subcaption}

%% metadata settings
\newcommand{\tutnum}{12}
\title{\tutnum. cvičení z diskrétní matematiky}
\author{Jan Hartman}
\date{5.1.2026}

\newcommand{\teacherurl}{https://kam.mff.cuni.cz/~hartmaj/}
\newcommand{\titlerule}{%
    \noindent %
    \makebox[\textwidth]{\large \thetitle \hfill \thedate}
    \rule{\textwidth}{0.4pt}%
}

%% headers and footers settings
\renewcommand{\headrulewidth}{0pt}
\renewcommand{\footrulewidth}{0.4pt}
\pagestyle{fancy}
\fancyhf{} % clear all header/footer (e.g. by default, footer would show page numbers)
% \fancyhead[L]{\thetitle}   % left -> title
% \fancyhead[C]{} % center -> unused
% \fancyhead[R]{\thedate}  % right -> date
\fancyfoot[C]{\small Více info k cvičení: \url{\teacherurl}}  

% Napady:
% Nejmensi graf t.z. splnuje eulerovu formuli, ale neni to strom
% Splňuje-li graf Eulerovu formuli, tak obsahuje list nebo izolovaný vrchol

% Definovat:
% strom
% kostra
% vnitrni vrchol, list

% Document body
\begin{document}

\titlerule

\begin{defn}[Klika]
Klikou grafu $G$ nazýváme jeho libovolný podgraf izomorfní úplnému grafu $K_n$ pro nějaké $n$.
\end{defn}

\begin{defn}[Klikovost]
Klikovost grafu $G$ je velikost největší kliky, kterou $G$ obsahuje jako podgraf.
\end{defn}

\begin{defn}[Krychle]
Definujeme $n$-dimenzionální krychli jako graf $Q_n$, jehož vrcholy odpovídají posloupnostem nul a jedniček délky $n$, tedy $V(Q_n)=\{0,1\}^n$, a mezi dvěma vrcholy $v,w$ vede hrana právě tehdy, když se posloupnosti $v$ a $w$ liší právě v jedné pozici.
\end{defn}

\begin{problem}[Princezniny korále]
Nechť $C_n$ je kružnice na $n$ vrcholech. Stanovte $\chi (C_n)$ v závislosti na $n$.
\end{problem}

\begin{problem}[Hledáme grafy na barvení]
Najděte příklad grafu, jehož barevnost je:
\begin{enumerate}[label=\alph*)]
    \item menší než jeho degenerovanost
    \item větší než jeho degenerovanost
    \item větší než jeho klikovost
\end{enumerate}
\end{problem}

\begin{problem}[Padesát odstínů zelené]
Nechť $T$ je strom. Určete chromatické číslo grafu $T$.
\end{problem}

\begin{problem}[Dokud je smrt nerozdělí]
Mějme dva grafy $G_1$ a $G_2$ s barevnostmi $\chi(G_1)$ a $\chi(G_2)$. Jaká bude barevnost grafu $H$, který vznikne slepením $G_1$ a $G_2$ za libovolný vrchol?
\end{problem}

\begin{problem}[Nestačí mi barvy]
Určete barevnost grafu níže a zdůvodněte:
\begin{figure}[h]
    \centering
    \includegraphics[width=100pt]{colorgraph.pdf}
\end{figure}
\end{problem}

\begin{problem}[Malý stupeň, mnoho radosti]
Dokažte, že každý rovinný graf má barevnost nejvýše $6$.
\end{problem}

\begin{problem}[Obarvení tisíckrát jinak]
Spočtěte, kolik existuje různých obarvení úplného grafu $K_n$ pomocí $k$ barev. Některé barvy mohou zůstat nevyužité. Obarvení $c_1$ a $c_2$ jsou různá, pokud existuje vrchol $v$ t.ž. $c_1(v) \neq c_2(v)$.
\end{problem}

\begin{problem}[Graf na zakázku se zákazem]
Nalezněte graf, jehož chromatické číslo je $4$, ale neobsahuje podgraf izomorfní $K_4$. 
\end{problem}

\begin{problem}[Krychle a zběsile]
Nechť $Q_d$ je $d$-dimenzionální krychle. Určete $\chi(Q_d)$ v závislosti na $d$.
\end{problem}

\begin{problem}[Trojúhelníky jsou tabu]
Buď $G$ rovinný graf, který neobsahuje $C_3$ jako podgraf, tedy $G$ je graf bez trojúhelníků. Dokažte, že $\chi(G) \leq 4$.
\end{problem}

\end{document}
