% Document class
\documentclass[10pt]{article}

% Preamble
\usepackage[a4paper,margin=2.5cm,includefoot]{geometry} % includefoot makes footer stay inside of the page

\usepackage{../../tutorial}
\usepackage{subcaption}

%% metadata settings
\newcommand{\tutnum}{8}
\title{\tutnum. cvičení z diskrétní matematiky}
\author{Jan Hartman}
\date{24.11.2025}

\newcommand{\teacherurl}{https://kam.mff.cuni.cz/~hartmaj/}
\newcommand{\titlerule}{%
    \noindent %
    \makebox[\textwidth]{\large \thetitle \hfill \thedate}
    \rule{\textwidth}{0.4pt}%
}

%% headers and footers settings
\renewcommand{\headrulewidth}{0pt}
\renewcommand{\footrulewidth}{0.4pt}
\pagestyle{fancy}
\fancyhf{} % clear all header/footer (e.g. by default, footer would show page numbers)
% \fancyhead[L]{\thetitle}   % left -> title
% \fancyhead[C]{} % center -> unused
% \fancyhead[R]{\thedate}  % right -> date
\fancyfoot[C]{\small Více info k cvičení: \url{\teacherurl}}  

% Document body
\begin{document}

\titlerule

\begin{defn}[Rozptyl]
Pro náhodnou veličinu $X$ definujeme její rozptyl jako $\Var(X) \coloneq \E[(X-\E[X])^2]$.
\end{defn}

\begin{theorem}[Markovova nerovnost]
Nechť $X$ je nezáporná náhodná veličina a $t > 0$. Potom platí: $\Pr(X \geq t) \leq \frac{\E[X]}{t}$.
\end{theorem}

\begin{problem}[Hazard se ne/vyplácí]
Podivná existence v kápi vám nabídla následující hazardní hru se dvěma férovými šestistěnnými kostkami: Nejprve si zvolíte přirozené číslo $n$. Následně zaplatíte jeden zlatý a hodíte kostkami. Padne-li na kostkách součet hodnot $n$, tak vyhráváte $n$ zlatých. V opačném případě nedostanete nic.
\begin{enumerate}[label=\alph*)]
    \item Popište pravděpodobnostní prostor, ve kterém se hra odehrává?
    \item Navrhněte náhodnou veličinu, která bude odpovídat vašemu získanému finančnímu obnosu, pokud za $n$ zvolíte $7$. Jaká je její střední hodnota?
    \item Najděte všechny možné hodnoty $n$, pro které se vám tato hra vyplatí.
\end{enumerate}


\end{problem}

\begin{problem}[Vánoční večírek]
Na vánoční večírek přišlo $n$ lidí a každý přinesl jeden dárek. Dárky náhodně rozdělíme mezi účastníky. Jaká je střední hodnota počtu lidí, kteří dostanou svůj vlastní dárek?
\end{problem}

\begin{problem}[Mince opět na scéně]
Úvahou určete, kolikrát je potřeba hodit férovou mincí tak, aby:
\begin{enumerate}[label=\alph*)]
    \item Střední hodnota počtu orlů byla $5$
    \item Pravděpodobnost, že padne alespoň $5$ orlů byla přesně $\frac{1}{2}$
\end{enumerate}
\end{problem}

\begin{problem}[Užitečný vzoreček]
Dokažte, že platí rovnost $\E[(X-\E[X])^2] = \E[X^2] - (\E[X])^2$
\end{problem}

\begin{problem}[Základní rozdělení]
Pro n.v. $X$ z následujících pravděpodobnostních rozdělení spočítejte pravděpodobnosti jednotlivých hodnot, její střední hodnotu a rozptyl:
\begin{enumerate}[label=\alph*)]
    \item \textit{Bernoulliho}: $X$ je $1$ s pravděpodobností $p$, jinak $0$
    \item \textit{Binomické}: $X$ je součet $n$ náhodných nezávislých veličin s Bernoulliho rozdělením
    \item \textit{Geometrické}: $X$ je počet Bernoulliho pokusů, dokud nedostaneme $1$
\end{enumerate}
\end{problem}

\begin{problem}[Zaječí úmysly]
Na palouku panáčkuje $n$ zajíců. Najednou se připlíží $n$ myslivců, každý z nich zamíří na jednoho náhodně vybraného zajíce a zastřelí ho.
\begin{enumerate}[label=\alph*)]
    \item Jaká je pravděpodobnost, že aspoň jeden zajíc přežije?
    \item Jaká je střední hodnota přeživších zajíců?
\end{enumerate}
\end{problem}

\begin{problem}[Promiňte, už se nevejdete]
Dopravní podnik na jedné autobusové lince spustil počítání cestujících. Spočítal, že při výjezdu z vybrané zastávky se v autobuse vyskytuje průměrně 8 cestujících. Autobus má kapacitu 40 lidí. Co můžeme říci o pravděpodobnosti, že po obsloužení této zastávky bude autobus plný (nebo dokonce přeplněný)?
\end{problem}

\begin{problem}[Nezávislost doplňků]
Dokažte, že pokud jsou jevy $A$ a $B$ nezávislé, tak jsou i jevy $A$ a $\overline{B} = \Omega \setminus B$ nezávislé.
\end{problem}

\begin{problem}[Boj o nezávislost]
Najděte na pravděpodobnostním prostoru $(\Omega,P)$, kde $\Omega \coloneq \{0,1\}^3$ a $\forall \omega \in \Omega: P(\omega)=\frac{1}{2^3}$ jevy $A,B,C$ pro které platí $P(A \cap B \cap C) = P(A) \cdot P(B) \cdot P(C)$, ale neplatí $P(A \cap B) = P(A) \cdot P(B)$.
\end{problem}

\end{document}