% Document class
\documentclass[10pt]{article}

% Preamble
\usepackage[a4paper,margin=2.5cm,includefoot]{geometry} % includefoot makes footer stay inside of the page

\usepackage{../../tutorial}
\usepackage{subcaption}

%% metadata settings
\newcommand{\tutnum}{11}
\title{\tutnum. cvičení z diskrétní matematiky}
\author{Jan Hartman}
\date{15.12.2025}

\newcommand{\teacherurl}{https://kam.mff.cuni.cz/~hartmaj/}
\newcommand{\titlerule}{%
    \noindent %
    \makebox[\textwidth]{\large \thetitle \hfill \thedate}
    \rule{\textwidth}{0.4pt}%
}

%% headers and footers settings
\renewcommand{\headrulewidth}{0pt}
\renewcommand{\footrulewidth}{0.4pt}
\pagestyle{fancy}
\fancyhf{} % clear all header/footer (e.g. by default, footer would show page numbers)
% \fancyhead[L]{\thetitle}   % left -> title
% \fancyhead[C]{} % center -> unused
% \fancyhead[R]{\thedate}  % right -> date
\fancyfoot[C]{\small Více info k cvičení: \url{\teacherurl}}  

% Napady:
% Nejmensi graf t.z. splnuje eulerovu formuli, ale neni to strom
% Splňuje-li graf Eulerovu formuli, tak obsahuje list nebo izolovaný vrchol

% Definovat:
% strom
% kostra
% vnitrni vrchol, list

% Document body
\begin{document}

\titlerule

\begin{problem}[Eulerodetektor]
U následujících grafů rozhodněte, zda jsou eulerovské:
\begin{enumerate}[label=\alph*)]
    \item pro libovolnou množinu $M$ graf $G$ s vrcholy $V = 2^M$ a množinou hran $E = \{ \{S,U\} : S \cap U = \emptyset \}$ 
    \item graf $G'$, který vznikne odebráním vrcholu $M$ z grafu $G$ (a všech hran s ním incidentních)
    \item souvislý graf $G$ s lichým počtem vrcholů, jehož doplněk má všechny stupně sudé
\end{enumerate}
\end{problem}

\begin{problem}[Jeden Euler stačí]
Najděte eulerovský graf, jehož doplněk je souvislý, ale není eulerovský.
\end{problem}

\begin{problem}[Graf v rozkladu]
Ukažte, že libovolný graf, jehož všechny vrcholy mají sudý stupeň, lze rozložit na hranově disjunktní sjednocení kružnic a izolovaných vrcholů. (kružnice jsou hranově disjunktní, pokud neobsahují společnou hranu)
\end{problem}

\hrule

\begin{defn}
$Q_n$ (n-rozměrná krychle) je graf s vrcholy $2^{[n]}$ a hrany vedou mezi těmi
podmnožinami $A, B$, kde $|A \,\triangle\, B| = 1$.
\end{defn}

\begin{problem}[Řekněme si to na rovinu]
Rozhodněte, zda jsou grafy na obrázku rovinné či nikoliv.
\end{problem}
\begin{figure}[h]
    \centering
    \includegraphics{planar_nonplanar.pdf}
\end{figure}

\begin{problem}[Dračí doupě]
Pro každé $k \geq 1$ určete, zda existuje $k$-regulární rovinný graf.

\noindent
\textit{Hint: Můžete například využít toho, že grafy odpovídající kostrám těles v prostoru jsou rovinné.}
\end{problem}

\begin{problem}[Nerovinnost tisíckrát jinak]
Bez použití Kuratowského věty dokažte, že $K_5$ není rovinný.
\end{problem}

\begin{problem}[Čtyřdimenzionální krychle]
Ukažte, že graf $Q_4$ odpovídající čtyřdimenzionální krychli není rovinný.
\end{problem}

\begin{problem}[Regularita není zadarmo]
Může existovat $3$-regulární graf, jehož všechny stěny budou šestiúhelníky?
A co pětiúhelníky?
\end{problem}

\begin{problem}[Buď ty nebo já]
Dokažte, že pro každý graf $G$ na jedenácti vrcholech je $G$ nebo $\overline{G}$ nerovinný.
\end{problem}

\begin{problem}[Není vás trochu moc?]
Dokažte, že v rovinném grafu nejvýše polovina jeho vrcholů může mít stupeň větší než $11$.
\end{problem}

\begin{problem}[Torus Makto]
Nakreslete $K_{3,3}$ na torus.
\end{problem}

\end{document}
