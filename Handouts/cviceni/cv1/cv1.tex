% Document class
\documentclass[10pt]{article}

% Preamble
\usepackage[a4paper,margin=2.5cm,includefoot]{geometry} % includefoot makes footer stay inside of the page
\usepackage{titling}
\usepackage{fancyhdr}
\usepackage[hidelinks]{hyperref}
\usepackage[czech]{babel}
\usepackage{amsthm}
\usepackage{mathtools}
\usepackage{enumitem} % for a,b,c enums
\usepackage{caption}

% FONT SPEC (for XeLaTeX) TO MAKE TROJA PRINTER WORK
\usepackage{fontspec}
\usepackage{unicode-math}

%% metadata settings
\newcommand{\tutnum}{1}
\title{\tutnum. cvičení z diskrétní matematiky}
\author{Jan Hartman}
\date{6.10.2025}

\newcommand{\teacherurl}{https://kam.mff.cuni.cz/~hartmaj/}
\newcommand{\titlerule}{%
    \noindent %
    \makebox[\textwidth]{\large \thetitle \hfill \thedate}
    \rule{\textwidth}{0.4pt}%
}

%% headers and footers settings
\renewcommand{\headrulewidth}{0pt}
\renewcommand{\footrulewidth}{0.4pt}
\pagestyle{fancy}
\fancyhf{} % clear all header/footer (e.g. by default, footer would show page numbers)
% \fancyhead[L]{\thetitle}   % left -> title
% \fancyhead[C]{} % center -> unused
% \fancyhead[R]{\thedate}  % right -> date
\fancyfoot[C]{\small Více info k cvičení: \url{\teacherurl}}  


% theorem styles
\newtheoremstyle{definitionstyle}{10pt}{10pt}{\normalfont}{}{\bfseries}{.}{ }{\thmname{#1}}
\newtheoremstyle{problemstyle}{10pt}{10pt}{\normalfont}{}{\bfseries}{.\newline}{ }{\thmname{#1}\thmnumber{ #2}\thmnote{ (#3)}}

% theorems
\theoremstyle{definitionstyle}
\newtheorem{defn}{Definice}
\theoremstyle{problemstyle}
\newtheorem{problem}{Příklad}

% Document body
\begin{document}

\titlerule

\section{Důkazy}

\begin{problem}[Neexistence cyklů]
O jisté množině států $E$ jsme zjistili, že splňujě:
\begin{enumerate}
    \item $\forall x \in E \ \forall y \in E: V(x,y) \Rightarrow \neg V(y,x)$
    \item $\forall x \in E \ \forall y \in E \  \forall z \in E : V(x,y) \wedge V(y,z) \Rightarrow V(x,z)$
\end{enumerate} 
\begin{enumerate}[label=(\alph*)]
    \item Znázorněte pomocí bodů a šipek, co říká podmínka číslo 2.
    \item Dokažte, že v $E$ neexistuje posloupnost států $x_0, \ldots , x_n$ t.ž. $V(x_0,x_1) \wedge V(x_1,x_2) \wedge \ldots \wedge  V(x_{n-1},x_n)$ kde $x_0 = x_n$.

\end{enumerate}
\end{problem}

\section{Hádanky}

\begin{problem}[Rovnoramenná váha]
Máme k dispozici rovnoramennou váhu a 9 mincí. Jedna z mincí je ovšem falešná, což se pozná tak, že je lehčí než ostatní mince, které váží všechny stejně. Na kolik nejméně vážení dokážeme zjistit, která z mincí je falešná?
\end{problem}

\end{document}