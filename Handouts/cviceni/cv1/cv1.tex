% Document class
\documentclass[10pt]{article}

% Preamble
\usepackage[a4paper,margin=2.5cm,includefoot]{geometry} % includefoot makes footer stay inside of the page

\usepackage{../../tutorial}

%% metadata settings
\newcommand{\tutnum}{1}
\title{\tutnum. cvičení z diskrétní matematiky}
\author{Jan Hartman}
\date{6.10.2025}

\newcommand{\teacherurl}{https://kam.mff.cuni.cz/~hartmaj/}
\newcommand{\titlerule}{%
    \noindent %
    \makebox[\textwidth]{\large \thetitle \hfill \thedate}
    \rule{\textwidth}{0.4pt}%
}

%% headers and footers settings
\renewcommand{\headrulewidth}{0pt}
\renewcommand{\footrulewidth}{0.4pt}
\pagestyle{fancy}
\fancyhf{} % clear all header/footer (e.g. by default, footer would show page numbers)
% \fancyhead[L]{\thetitle}   % left -> title
% \fancyhead[C]{} % center -> unused
% \fancyhead[R]{\thedate}  % right -> date
\fancyfoot[C]{\small Více info k cvičení: \url{\teacherurl}}  


% Document body
\begin{document}

\titlerule

\begin{defn}
    Píšeme $n \ | \ m$ když existuje přirozené číslo $k$ takové, že $n \cdot k = m$.
\end{defn}

\begin{defn}
    Definujeme Fibonacciho posloupnost jako $F_0 = 0$, $F_1 = 1$ a pro každé $i \geq 2 : F_i = F_{i-1} + F_{i-2}$.
\end{defn}

\section{Důkazy}

\begin{problem}[Suma mocnin dvojky]
Dokažte, že pro každé $n \geq 0$ platí $\sum_{i=0}^{n}2^i = 2^{n+1} - 1$.
\end{problem}

\begin{problem}[Dělitelnost]
Indukcí dokažte, že pro každé $n \geq 0$ platí $4 \ | \ (6n^2 + 2n)$.
\end{problem}

\begin{problem}[Šachovnice]
Mějme šachovnici o rozměru $2^n \times 2^n$, ve které chybí právě jedno libovolné políčko. Dokažte, že lze zcela pokrýt kostičkami tvaru písmene L (zabírají tři políčka).
\end{problem}

\begin{problem}[Teleskopický součin]
Dokažte, že $\prod_{i=1}^{n}\frac{i+1}{i} = n + 1$ pro každé $n \geq 1$.
\end{problem}

\begin{problem}[Neexistence cyklů]
O jisté množině států $E$ jsme zjistili, že splňujě:
\begin{enumerate}
    \item $\forall x \in E \ \forall y \in E: V(x,y) \Rightarrow \neg V(y,x)$
    \item $\forall x \in E \ \forall y \in E \  \forall z \in E : V(x,y) \wedge V(y,z) \Rightarrow V(x,z)$
\end{enumerate} 
Dokažte, že v $E$ neexistuje posloupnost států $x_0, \ldots , x_n$ t.ž. $V(x_0,x_1) \wedge V(x_1,x_2) \wedge \ldots \wedge  V(x_{n-1},x_n)$ kde $x_0 = x_n$.
\end{problem}

\section{Hádanky}

\begin{problem}[Rovnoramenná váha]
Máme k dispozici rovnoramennou váhu a 9 mincí. Jedna z mincí je ovšem falešná, což se pozná tak, že je lehčí než ostatní mince, které váží všechny stejně. Na kolik nejméně vážení dokážeme zjistit, která z mincí je falešná?
\end{problem}

\end{document}