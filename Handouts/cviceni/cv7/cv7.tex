% Document class
\documentclass[10pt]{article}

% Preamble
\usepackage[a4paper,margin=2.5cm,includefoot]{geometry} % includefoot makes footer stay inside of the page

\usepackage{../../tutorial}
\usepackage{subcaption}

%% metadata settings
\newcommand{\tutnum}{7}
\title{\tutnum. cvičení z diskrétní matematiky}
\author{Jan Hartman}
\date{17.11.2025}

\newcommand{\teacherurl}{https://kam.mff.cuni.cz/~hartmaj/}
\newcommand{\titlerule}{%
    \noindent %
    \makebox[\textwidth]{\large \thetitle \hfill \thedate}
    \rule{\textwidth}{0.4pt}%
}

%% headers and footers settings
\renewcommand{\headrulewidth}{0pt}
\renewcommand{\footrulewidth}{0.4pt}
\pagestyle{fancy}
\fancyhf{} % clear all header/footer (e.g. by default, footer would show page numbers)
% \fancyhead[L]{\thetitle}   % left -> title
% \fancyhead[C]{} % center -> unused
% \fancyhead[R]{\thedate}  % right -> date
\fancyfoot[C]{\small Více info k cvičení: \url{\teacherurl}}  

% Document body
\begin{document}

\titlerule

\begin{problem}[Gaudeamus igitur]

Nechť $P$ je následující píseň:

\begin{figure}[h]
    \includegraphics[width=\textwidth]{gaudeamus_igitur.png}
\end{figure}



\begin{enumerate}[label=\alph*)]
    \item Naučte se hrát na klavír a zahrajte si melodii $P$
    \item Naučte se zpívat a zazpívejte si text $P$
    \item Naučte se latinsky a zjistěte, o čem $P$ je
    \item Zjistěte, jak souvisí $P$ s Mezinárodním dnem studentstva
\end{enumerate}

\end{problem}

\begin{problem}[Demonstremus igitur]
Dokažte nebo vyvraťte: „Každé sudé číslo větší než $2$ lze vyjádřit jako součet dvou prvočísel.“
\end{problem}

\vspace{5cm}
\begin{center}
\fbox{
    \begin{minipage}{0.8\textwidth}
        \begin{center}
            \textbf{\Large KDY -- KDYŽ NE TEĎ?}\\[6pt]
            \textbf{\Large KDO -- KDYŽ NE MY?}\\[10pt]
            \textbf{\large 17.\,11.\,1989}
        \end{center}
    \end{minipage}
}
\end{center}

\end{document}