% Document class
\documentclass[10pt]{article}

% Preamble
\usepackage[a5paper,margin=1cm,includefoot]{geometry} % includefoot makes footer stay inside of the page
\usepackage{titling}
\usepackage{fancyhdr}
\usepackage[hidelinks]{hyperref}
\usepackage{amsthm}
\usepackage{mathtools}

%% metadata settings
\newcommand{\tutnum}{0}
\title{\tutnum. cvičení z diskrétní matematiky}
\author{Jan Hartman}
\date{29.9.2025}

\newcommand{\teacherurl}{https://kam.mff.cuni.cz/~hartmaj/}
\newcommand{\titlerule}{%
    \noindent %
    \makebox[\textwidth]{\large \thetitle \hfill \thedate}
    \rule{\textwidth}{0.4pt}%
}


%% headers and footers settings
\renewcommand{\headrulewidth}{0pt}
\renewcommand{\footrulewidth}{0.4pt}
\pagestyle{fancy}
\fancyhf{} % clear all header/footer (e.g. by default, footer would show page numbers)
% \fancyhead[L]{\thetitle}   % left -> title
% \fancyhead[C]{} % center -> unused
% \fancyhead[R]{\thedate}  % right -> date
\fancyfoot[C]{\small Více info k cvičení: \url{\teacherurl}}  


% theorem styles
\newtheoremstyle{definitionstyle}{10pt}{10pt}{\normalfont}{}{\bfseries}{.}{ }{\thmname{#1}}
\newtheoremstyle{problemstyle}{10pt}{10pt}{\normalfont}{}{\bfseries}{.\newline}{ }{\thmname{#1}\thmnumber{ #2}\thmnote{ (#3)}}

% theorems
\theoremstyle{definitionstyle}
\newtheorem{defn}{Definice}
\theoremstyle{problemstyle}
\newtheorem{problem}{Příklad}

% Document body
\begin{document}

\titlerule

\begin{defn}[Fibonacciho číslo]
Definujeme Fibonacciho čísla následovně: $F_1 \coloneq 0$, $F_2 \coloneq 1$ a pro $n > 2$ je $F_n \coloneq F_{n-1}+F_{n-2}$
\end{defn}

\begin{problem}[Rovnoramenná váha]
Máme k dispozici rovnoramennou váhu a 9 mincí. Jedna z mincí je ovšem falešná, což se pozná tak, že je lehčí než ostatní mince, které váží všechny stejně. Na kolik nejméně vážení dokážeme zjistit, která z mincí je falešná?
\end{problem}

\begin{problem}[Topinky]
Jakožto chudý student nemající dost peněz na topinkovač si smažíme topinky na pánvi. Opéci jednu stranu topinky trvá 2 minuty. Na pánev se vejdou současně nejvýš dva krajíce. Jak dlouho bude trvat opečení 3 krajíců chleba?
\end{problem}

\end{document}